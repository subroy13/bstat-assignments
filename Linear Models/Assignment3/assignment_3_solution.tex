\documentclass[]{article}
\usepackage{lmodern}
\usepackage{float}
\usepackage{amssymb,amsmath}
\usepackage{ifxetex,ifluatex}
\usepackage{fixltx2e} % provides \textsubscript
\ifnum 0\ifxetex 1\fi\ifluatex 1\fi=0 % if pdftex
  \usepackage[T1]{fontenc}
  \usepackage[utf8]{inputenc}
\else % if luatex or xelatex
  \ifxetex
    \usepackage{mathspec}
  \else
    \usepackage{fontspec}
  \fi
  \defaultfontfeatures{Ligatures=TeX,Scale=MatchLowercase}
\fi
% use upquote if available, for straight quotes in verbatim environments
\IfFileExists{upquote.sty}{\usepackage{upquote}}{}
% use microtype if available
\IfFileExists{microtype.sty}{%
\usepackage{microtype}
\UseMicrotypeSet[protrusion]{basicmath} % disable protrusion for tt fonts
}{}
\usepackage[margin=1in]{geometry}
\usepackage{hyperref}
\hypersetup{unicode=true,
            pdftitle={Linear Models Assignment 3},
            pdfauthor={Subhrajyoty Roy (BS-1613)},
            pdfborder={0 0 0},
            breaklinks=true}
\urlstyle{same}  % don't use monospace font for urls
\usepackage{color}
\usepackage{fancyvrb}
\newcommand{\VerbBar}{|}
\newcommand{\VERB}{\Verb[commandchars=\\\{\}]}
\DefineVerbatimEnvironment{Highlighting}{Verbatim}{commandchars=\\\{\}}
% Add ',fontsize=\small' for more characters per line
\usepackage{framed}
\definecolor{shadecolor}{RGB}{248,248,248}
\newenvironment{Shaded}{\begin{snugshade}}{\end{snugshade}}
\newcommand{\KeywordTok}[1]{\textcolor[rgb]{0.13,0.29,0.53}{\textbf{#1}}}
\newcommand{\DataTypeTok}[1]{\textcolor[rgb]{0.13,0.29,0.53}{#1}}
\newcommand{\DecValTok}[1]{\textcolor[rgb]{0.00,0.00,0.81}{#1}}
\newcommand{\BaseNTok}[1]{\textcolor[rgb]{0.00,0.00,0.81}{#1}}
\newcommand{\FloatTok}[1]{\textcolor[rgb]{0.00,0.00,0.81}{#1}}
\newcommand{\ConstantTok}[1]{\textcolor[rgb]{0.00,0.00,0.00}{#1}}
\newcommand{\CharTok}[1]{\textcolor[rgb]{0.31,0.60,0.02}{#1}}
\newcommand{\SpecialCharTok}[1]{\textcolor[rgb]{0.00,0.00,0.00}{#1}}
\newcommand{\StringTok}[1]{\textcolor[rgb]{0.31,0.60,0.02}{#1}}
\newcommand{\VerbatimStringTok}[1]{\textcolor[rgb]{0.31,0.60,0.02}{#1}}
\newcommand{\SpecialStringTok}[1]{\textcolor[rgb]{0.31,0.60,0.02}{#1}}
\newcommand{\ImportTok}[1]{#1}
\newcommand{\CommentTok}[1]{\textcolor[rgb]{0.56,0.35,0.01}{\textit{#1}}}
\newcommand{\DocumentationTok}[1]{\textcolor[rgb]{0.56,0.35,0.01}{\textbf{\textit{#1}}}}
\newcommand{\AnnotationTok}[1]{\textcolor[rgb]{0.56,0.35,0.01}{\textbf{\textit{#1}}}}
\newcommand{\CommentVarTok}[1]{\textcolor[rgb]{0.56,0.35,0.01}{\textbf{\textit{#1}}}}
\newcommand{\OtherTok}[1]{\textcolor[rgb]{0.56,0.35,0.01}{#1}}
\newcommand{\FunctionTok}[1]{\textcolor[rgb]{0.00,0.00,0.00}{#1}}
\newcommand{\VariableTok}[1]{\textcolor[rgb]{0.00,0.00,0.00}{#1}}
\newcommand{\ControlFlowTok}[1]{\textcolor[rgb]{0.13,0.29,0.53}{\textbf{#1}}}
\newcommand{\OperatorTok}[1]{\textcolor[rgb]{0.81,0.36,0.00}{\textbf{#1}}}
\newcommand{\BuiltInTok}[1]{#1}
\newcommand{\ExtensionTok}[1]{#1}
\newcommand{\PreprocessorTok}[1]{\textcolor[rgb]{0.56,0.35,0.01}{\textit{#1}}}
\newcommand{\AttributeTok}[1]{\textcolor[rgb]{0.77,0.63,0.00}{#1}}
\newcommand{\RegionMarkerTok}[1]{#1}
\newcommand{\InformationTok}[1]{\textcolor[rgb]{0.56,0.35,0.01}{\textbf{\textit{#1}}}}
\newcommand{\WarningTok}[1]{\textcolor[rgb]{0.56,0.35,0.01}{\textbf{\textit{#1}}}}
\newcommand{\AlertTok}[1]{\textcolor[rgb]{0.94,0.16,0.16}{#1}}
\newcommand{\ErrorTok}[1]{\textcolor[rgb]{0.64,0.00,0.00}{\textbf{#1}}}
\newcommand{\NormalTok}[1]{#1}
\usepackage{graphicx,grffile}
\makeatletter
\def\maxwidth{\ifdim\Gin@nat@width>\linewidth\linewidth\else\Gin@nat@width\fi}
\def\maxheight{\ifdim\Gin@nat@height>\textheight\textheight\else\Gin@nat@height\fi}
\makeatother
% Scale images if necessary, so that they will not overflow the page
% margins by default, and it is still possible to overwrite the defaults
% using explicit options in \includegraphics[width, height, ...]{}
\setkeys{Gin}{width=\maxwidth,height=\maxheight,keepaspectratio}
\IfFileExists{parskip.sty}{%
\usepackage{parskip}
}{% else
\setlength{\parindent}{0pt}
\setlength{\parskip}{6pt plus 2pt minus 1pt}
}
\setlength{\emergencystretch}{3em}  % prevent overfull lines
\providecommand{\tightlist}{%
  \setlength{\itemsep}{0pt}\setlength{\parskip}{0pt}}
\setcounter{secnumdepth}{0}
% Redefines (sub)paragraphs to behave more like sections
\ifx\paragraph\undefined\else
\let\oldparagraph\paragraph
\renewcommand{\paragraph}[1]{\oldparagraph{#1}\mbox{}}
\fi
\ifx\subparagraph\undefined\else
\let\oldsubparagraph\subparagraph
\renewcommand{\subparagraph}[1]{\oldsubparagraph{#1}\mbox{}}
\fi

%%% Use protect on footnotes to avoid problems with footnotes in titles
\let\rmarkdownfootnote\footnote%
\def\footnote{\protect\rmarkdownfootnote}

%%% Change title format to be more compact
\usepackage{titling}

% Create subtitle command for use in maketitle
\newcommand{\subtitle}[1]{
  \posttitle{
    \begin{center}\large#1\end{center}
    }
}

\setlength{\droptitle}{-2em}

  \title{Linear Models Assignment 3}
    \pretitle{\vspace{\droptitle}\centering\huge}
  \posttitle{\par}
    \author{Subhrajyoty Roy (BS-1613)}
    \preauthor{\centering\large\emph}
  \postauthor{\par}
      \predate{\centering\large\emph}
  \postdate{\par}
    \date{October 25, 2018}


\begin{document}
\maketitle

\section{Loading the Prequisites
Libraries}\label{loading-the-prequisites-libraries}

The package \emph{reshape2} has been used in order to mainpulate the
structure of the data and transform it into the required format. I have
mainly used \emph{DescTools} package to obtain the multiple comparison
tests. However, there is no support for bonferroni test for arbitrary
contrasts in \emph{DescTools} package, so I need to use \emph{multcomp}
package for this purpose. The guide is given in the url:
\url{https://stat.ethz.ch/~meier/teaching/anova/contrasts-and-multiple-testing.html}

\begin{Shaded}
\begin{Highlighting}[]
\KeywordTok{library}\NormalTok{(reshape2)}
\KeywordTok{library}\NormalTok{(DescTools)}
\KeywordTok{library}\NormalTok{(multcomp)}
\end{Highlighting}
\end{Shaded}

I have used the packages \emph{rmarkdown}, \emph{knitr} and
\emph{kableExtra} in order to generate the report in .pdf format.

\begin{Shaded}
\begin{Highlighting}[]
\KeywordTok{library}\NormalTok{(knitr)}
\KeywordTok{library}\NormalTok{(kableExtra)}
\end{Highlighting}
\end{Shaded}

\section{Question 1}\label{question-1}

\textbf{Consider problem 1 of your Assignment 1. For each program, you
delete two observations and thus it is a balanced one way model.}

\begin{Shaded}
\begin{Highlighting}[]
\NormalTok{firmdata <-}\StringTok{ }\KeywordTok{data.frame}\NormalTok{(}\DataTypeTok{prog1 =} \KeywordTok{c}\NormalTok{(}\DecValTok{9}\NormalTok{,}\DecValTok{12}\NormalTok{,}\DecValTok{14}\NormalTok{,}\DecValTok{11}\NormalTok{,}\DecValTok{13}\NormalTok{), }\DataTypeTok{prog2 =} \KeywordTok{c}\NormalTok{(}\DecValTok{10}\NormalTok{,}\DecValTok{6}\NormalTok{,}\DecValTok{9}\NormalTok{,}\DecValTok{9}\NormalTok{,}\DecValTok{10}\NormalTok{),}
                       \DataTypeTok{prog3 =} \KeywordTok{c}\NormalTok{(}\DecValTok{12}\NormalTok{,}\DecValTok{14}\NormalTok{,}\DecValTok{11}\NormalTok{,}\DecValTok{13}\NormalTok{,}\DecValTok{11}\NormalTok{), }\DataTypeTok{prog4 =} \KeywordTok{c}\NormalTok{(}\DecValTok{9}\NormalTok{,}\DecValTok{8}\NormalTok{,}\DecValTok{11}\NormalTok{,}\DecValTok{7}\NormalTok{,}\DecValTok{8}\NormalTok{))}
\KeywordTok{kable}\NormalTok{(firmdata) }\OperatorTok\StringTok{ }\KeywordTok{kable_styling}\NormalTok{(}\DataTypeTok{full_width =}\NormalTok{ F)}
\end{Highlighting}
\end{Shaded}

\begin{table}[H]
\centering
\begin{tabular}{r|r|r|r}
\hline
prog1 & prog2 & prog3 & prog4\\
\hline
9 & 10 & 12 & 9\\
\hline
12 & 6 & 14 & 8\\
\hline
14 & 9 & 11 & 11\\
\hline
11 & 9 & 13 & 7\\
\hline
13 & 10 & 11 & 8\\
\hline
\end{tabular}
\end{table}

Now, I have to delete two observations for each program.

\begin{Shaded}
\begin{Highlighting}[]
\KeywordTok{set.seed}\NormalTok{(}\DecValTok{1613}\NormalTok{)  }\CommentTok{#this is my roll number, set the seed for reproducbility}
\NormalTok{x =}\StringTok{ }\KeywordTok{sample}\NormalTok{(}\DecValTok{1}\OperatorTok{:}\DecValTok{5}\NormalTok{, }\DataTypeTok{size =} \DecValTok{2}\NormalTok{)}
\NormalTok{firmdata =}\StringTok{ }\NormalTok{firmdata[}\OperatorTok{-}\NormalTok{x,]  }\CommentTok{#remove the corresponding observations}
\NormalTok{firmdata =}\StringTok{ }\KeywordTok{melt}\NormalTok{(firmdata)}
\NormalTok{firmdata}\OperatorTok{$}\NormalTok{variable =}\StringTok{ }\KeywordTok{as.factor}\NormalTok{(firmdata}\OperatorTok{$}\NormalTok{variable)}
\KeywordTok{kable}\NormalTok{(firmdata) }\OperatorTok\StringTok{ }\KeywordTok{kable_styling}\NormalTok{(}\DataTypeTok{full_width =}\NormalTok{ F)}
\end{Highlighting}
\end{Shaded}

\begin{table}[H]
\centering
\begin{tabular}{l|r}
\hline
variable & value\\
\hline
prog1 & 9\\
\hline
prog1 & 12\\
\hline
prog1 & 13\\
\hline
prog2 & 10\\
\hline
prog2 & 6\\
\hline
prog2 & 10\\
\hline
prog3 & 12\\
\hline
prog3 & 14\\
\hline
prog3 & 11\\
\hline
prog4 & 9\\
\hline
prog4 & 8\\
\hline
prog4 & 8\\
\hline
\end{tabular}
\end{table}

\textbf{Let the effect of Program i be denoted by \(\alpha_i\). Let the
set S1 denote the collection of elementary contrasts of the form
\(\alpha_i-\alpha_j, i<j\). Let S2 denote the collection of general
contrasts of the form \(\alpha_1 - 2\alpha_2 + \alpha_3\) and
\(2\alpha_1 - \alpha_2 - \alpha_3\). Let S3 be the union of S2 and S1.
Compute simultaneous confidence intervals with 95\% confidence
coefficient for S1 with Scheffe, Tukey and Bonferroni's method.}

\begin{Shaded}
\begin{Highlighting}[]
\CommentTok{#fits the ANOVA model}
\NormalTok{model =}\StringTok{ }\KeywordTok{aov}\NormalTok{(value }\OperatorTok{~}\StringTok{ }\NormalTok{variable, firmdata)}

\CommentTok{#calculates the Tukey HSD}
\NormalTok{res =}\StringTok{ }\KeywordTok{PostHocTest}\NormalTok{(model, }\DataTypeTok{method =} \StringTok{"hsd"}\NormalTok{, }\DataTypeTok{conf.level =} \FloatTok{0.95}\NormalTok{)  }
\NormalTok{res[[}\DecValTok{1}\NormalTok{]] =}\StringTok{ }\KeywordTok{cbind}\NormalTok{(res[[}\DecValTok{1}\NormalTok{]], }\DataTypeTok{length =}\NormalTok{ (res[[}\DecValTok{1}\NormalTok{]][,}\StringTok{"upr.ci"}\NormalTok{] }\OperatorTok{-}\StringTok{ }\NormalTok{res[[}\DecValTok{1}\NormalTok{]][,}\StringTok{"lwr.ci"}\NormalTok{])) }
\CommentTok{#create a new column containing length of the intervals}
\KeywordTok{print}\NormalTok{(res)}
\end{Highlighting}
\end{Shaded}

\begin{verbatim}

  Posthoc multiple comparisons of means : Tukey HSD 
    95% family-wise confidence level

$variable
                  diff     lwr.ci    upr.ci   pval   length    
prog2-prog1 -2.6666667 -7.2579456 1.9246123 0.3155 9.182558    
prog3-prog1  1.0000000 -3.5912789 5.5912789 0.8952 9.182558    
prog4-prog1 -3.0000000 -7.5912789 1.5912789 0.2340 9.182558    
prog3-prog2  3.6666667 -0.9246123 8.2579456 0.1241 9.182558    
prog4-prog2 -0.3333333 -4.9246123 4.2579456 0.9952 9.182558    
prog4-prog3 -4.0000000 -8.5912789 0.5912789 0.0895 9.182558 .  

---
Signif. codes:  0 '***' 0.001 '**' 0.01 '*' 0.05 '.' 0.1 ' ' 1
\end{verbatim}

\begin{Shaded}
\begin{Highlighting}[]
\CommentTok{#calculates the Scheffe's test}
\NormalTok{res =}\StringTok{ }\KeywordTok{PostHocTest}\NormalTok{(model, }\DataTypeTok{method =} \StringTok{"scheffe"}\NormalTok{, }\DataTypeTok{conf.level =} \FloatTok{0.95}\NormalTok{)  }
\NormalTok{res[[}\DecValTok{1}\NormalTok{]] =}\StringTok{ }\KeywordTok{cbind}\NormalTok{(res[[}\DecValTok{1}\NormalTok{]], }\DataTypeTok{length =}\NormalTok{ (res[[}\DecValTok{1}\NormalTok{]][,}\StringTok{"upr.ci"}\NormalTok{] }\OperatorTok{-}\StringTok{ }\NormalTok{res[[}\DecValTok{1}\NormalTok{]][,}\StringTok{"lwr.ci"}\NormalTok{]))}
\KeywordTok{print}\NormalTok{(res)}
\end{Highlighting}
\end{Shaded}

\begin{verbatim}

  Posthoc multiple comparisons of means : Scheffe Test 
    95% family-wise confidence level

$variable
                  diff    lwr.ci   upr.ci   pval   length    
prog2-prog1 -2.6666667 -7.674139 2.340806 0.3854 10.01494    
prog3-prog1  1.0000000 -4.007472 6.007472 0.9188 10.01494    
prog4-prog1 -3.0000000 -8.007472 2.007472 0.2967 10.01494    
prog3-prog2  3.6666667 -1.340806 8.674139 0.1682 10.01494    
prog4-prog2 -0.3333333 -5.340806 4.674139 0.9964 10.01494    
prog4-prog3 -4.0000000 -9.007472 1.007472 0.1249 10.01494    

---
Signif. codes:  0 '***' 0.001 '**' 0.01 '*' 0.05 '.' 0.1 ' ' 1
\end{verbatim}

\begin{Shaded}
\begin{Highlighting}[]
\CommentTok{#calculates the Bonferroni's test}
\NormalTok{res =}\StringTok{ }\KeywordTok{PostHocTest}\NormalTok{(model, }\DataTypeTok{method =} \StringTok{"bonferroni"}\NormalTok{, }\DataTypeTok{conf.level =} \FloatTok{0.95}\NormalTok{)  }
\NormalTok{res[[}\DecValTok{1}\NormalTok{]] =}\StringTok{ }\KeywordTok{cbind}\NormalTok{(res[[}\DecValTok{1}\NormalTok{]], }\DataTypeTok{length =}\NormalTok{ (res[[}\DecValTok{1}\NormalTok{]][,}\StringTok{"upr.ci"}\NormalTok{] }\OperatorTok{-}\StringTok{ }\NormalTok{res[[}\DecValTok{1}\NormalTok{]][,}\StringTok{"lwr.ci"}\NormalTok{]))}
\KeywordTok{print}\NormalTok{(res)}
\end{Highlighting}
\end{Shaded}

\begin{verbatim}

  Posthoc multiple comparisons of means : Bonferroni 
    95% family-wise confidence level

$variable
                  diff    lwr.ci    upr.ci   pval   length    
prog2-prog1 -2.6666667 -7.654408 2.3210751 0.5996 9.975483    
prog3-prog1  1.0000000 -3.987742 5.9877417 1.0000 9.975483    
prog4-prog1 -3.0000000 -7.987742 1.9877417 0.4185 9.975483    
prog3-prog2  3.6666667 -1.321075 8.6544084 0.2027 9.975483    
prog4-prog2 -0.3333333 -5.321075 4.6544084 1.0000 9.975483    
prog4-prog3 -4.0000000 -8.987742 0.9877417 0.1414 9.975483    

---
Signif. codes:  0 '***' 0.001 '**' 0.01 '*' 0.05 '.' 0.1 ' ' 1
\end{verbatim}

\textbf{For S2 and S3 obtain simultaneous confidence intervals using
Scheffe's method as well as Bonferroni's method.}

Firstly, the matrix containing the coefficients for each of the
contrasts is created.

\begin{Shaded}
\begin{Highlighting}[]
\NormalTok{s2 =}\StringTok{ }\KeywordTok{Permn}\NormalTok{(}\KeywordTok{c}\NormalTok{(}\DecValTok{0}\NormalTok{,}\OperatorTok{-}\DecValTok{1}\NormalTok{,}\OperatorTok{-}\DecValTok{1}\NormalTok{,}\DecValTok{2}\NormalTok{))  }\CommentTok{#computes all permutations of the coefficients}
\KeywordTok{colnames}\NormalTok{(s2) <-}\KeywordTok{paste}\NormalTok{(}\StringTok{"Coefficient of prog"}\NormalTok{, }\DecValTok{1}\OperatorTok{:}\DecValTok{4}\NormalTok{)}
\KeywordTok{kable}\NormalTok{(s2) }\OperatorTok\StringTok{ }\KeywordTok{kable_styling}\NormalTok{(}\DataTypeTok{full_width =}\NormalTok{ F)}
\end{Highlighting}
\end{Shaded}

\begin{table}[H]
\centering
\begin{tabular}{r|r|r|r}
\hline
Coefficient of prog 1 & Coefficient of prog 2 & Coefficient of prog 3 & Coefficient of prog 4\\
\hline
0 & -1 & -1 & 2\\
\hline
-1 & 0 & -1 & 2\\
\hline
-1 & -1 & 0 & 2\\
\hline
-1 & -1 & 2 & 0\\
\hline
0 & -1 & 2 & -1\\
\hline
-1 & 0 & 2 & -1\\
\hline
-1 & 2 & 0 & -1\\
\hline
-1 & 2 & -1 & 0\\
\hline
0 & 2 & -1 & -1\\
\hline
2 & 0 & -1 & -1\\
\hline
2 & -1 & 0 & -1\\
\hline
2 & -1 & -1 & 0\\
\hline
\end{tabular}
\end{table}

I obtain simulteneous confidence intervals for the above set of
contrasts in S2 using the method of Scheffe.

\begin{Shaded}
\begin{Highlighting}[]
\NormalTok{res =}\StringTok{ }\KeywordTok{ScheffeTest}\NormalTok{(model, }\DataTypeTok{conf.level =} \FloatTok{0.95}\NormalTok{, }\DataTypeTok{contrasts =} \KeywordTok{t}\NormalTok{(s2))}
\NormalTok{res[[}\DecValTok{1}\NormalTok{]] =}\StringTok{ }\KeywordTok{cbind}\NormalTok{(res[[}\DecValTok{1}\NormalTok{]], }\DataTypeTok{length =}\NormalTok{ (res[[}\DecValTok{1}\NormalTok{]][,}\StringTok{"upr.ci"}\NormalTok{] }\OperatorTok{-}\StringTok{ }\NormalTok{res[[}\DecValTok{1}\NormalTok{]][,}\StringTok{"lwr.ci"}\NormalTok{]))}
\KeywordTok{print}\NormalTok{(res)}
\end{Highlighting}
\end{Shaded}

\begin{verbatim}

  Posthoc multiple comparisons of means : Scheffe Test 
    95% family-wise confidence level

$variable
                       diff     lwr.ci    upr.ci   pval   length    
prog4-prog2,prog3 -4.333333 -13.006530  4.339863 0.4353 17.34639    
prog4-prog1,prog3 -7.000000 -15.673197  1.673197 0.1204 17.34639    
prog4-prog1,prog2 -3.333333 -12.006530  5.339863 0.6325 17.34639    
prog3-prog1,prog2  4.666667  -4.006530 13.339863 0.3774 17.34639    
prog3-prog2,prog4  7.666667  -1.006530 16.339863 0.0849 17.34639 .  
prog3-prog1,prog4  5.000000  -3.673197 13.673197 0.3250 17.34639    
prog2-prog1,prog4 -2.333333 -11.006530  6.339863 0.8286 17.34639    
prog2-prog1,prog3 -6.333333 -15.006530  2.339863 0.1697 17.34639    
prog2-prog3,prog4 -3.333333 -12.006530  5.339863 0.6325 17.34639    
prog1-prog3,prog4  2.000000  -6.673197 10.673197 0.8824 17.34639    
prog1-prog2,prog4  5.666667  -3.006530 14.339863 0.2368 17.34639    
prog1-prog2,prog3  1.666667  -7.006530 10.339863 0.9267 17.34639    

---
Signif. codes:  0 '***' 0.001 '**' 0.01 '*' 0.05 '.' 0.1 ' ' 1
\end{verbatim}

I obtain simulteneous confidence intervals for the same set of contrasts
using the method of Bonferroni.

\begin{Shaded}
\begin{Highlighting}[]
\NormalTok{model.gh =}\StringTok{ }\KeywordTok{glht}\NormalTok{(model, }\DataTypeTok{linfct =} \KeywordTok{mcp}\NormalTok{(}\DataTypeTok{variable =}\NormalTok{ s2))  }\CommentTok{#set the contrasts}
\NormalTok{res.gh =}\StringTok{ }\KeywordTok{confint}\NormalTok{(model.gh, }\DataTypeTok{test =} \KeywordTok{adjusted}\NormalTok{(}\StringTok{"bonferroni"}\NormalTok{))}
\KeywordTok{print}\NormalTok{(res.gh)}
\end{Highlighting}
\end{Shaded}

\begin{verbatim}

     Simultaneous Confidence Intervals

Multiple Comparisons of Means: User-defined Contrasts


Fit: aov(formula = value ~ variable, data = firmdata)

Quantile = 3.3277
95% family-wise confidence level
 

Linear Hypotheses:
        Estimate lwr      upr     
1 == 0   -4.3333 -12.5968   3.9302
2 == 0   -7.0000 -15.2635   1.2635
3 == 0   -3.3333 -11.5968   4.9302
4 == 0    4.6667  -3.5968  12.9302
5 == 0    7.6667  -0.5968  15.9302
6 == 0    5.0000  -3.2635  13.2635
7 == 0   -2.3333 -10.5968   5.9302
8 == 0   -6.3333 -14.5968   1.9302
9 == 0   -3.3333 -11.5968   4.9302
10 == 0   2.0000  -6.2635  10.2635
11 == 0   5.6667  -2.5968  13.9302
12 == 0   1.6667  -6.5968   9.9302
\end{verbatim}

\begin{Shaded}
\begin{Highlighting}[]
\NormalTok{res.gh}\OperatorTok{$}\NormalTok{confint[,}\StringTok{"upr"}\NormalTok{] }\OperatorTok{-}\StringTok{ }\NormalTok{res.gh}\OperatorTok{$}\NormalTok{confint[,}\StringTok{"lwr"}\NormalTok{] }\CommentTok{#compute the lengths of CI's}
\end{Highlighting}
\end{Shaded}

\begin{verbatim}
     1      2      3      4      5      6      7      8      9     10 
16.527 16.527 16.527 16.527 16.527 16.527 16.527 16.527 16.527 16.527 
    11     12 
16.527 16.527 
\end{verbatim}

Now, I have performed the similar for the set of contrasts in s3. There
would be (12+6) = 18 many contrasts in s3.

\begin{Shaded}
\begin{Highlighting}[]
\CommentTok{#computes the 6 pairwise contrasts}
\NormalTok{s1 <-}\StringTok{ }\KeywordTok{rbind}\NormalTok{(}\KeywordTok{c}\NormalTok{(}\OperatorTok{-}\DecValTok{1}\NormalTok{,}\DecValTok{1}\NormalTok{,}\DecValTok{0}\NormalTok{,}\DecValTok{0}\NormalTok{), }\KeywordTok{c}\NormalTok{(}\OperatorTok{-}\DecValTok{1}\NormalTok{,}\DecValTok{0}\NormalTok{,}\DecValTok{1}\NormalTok{,}\DecValTok{0}\NormalTok{), }\KeywordTok{c}\NormalTok{(}\OperatorTok{-}\DecValTok{1}\NormalTok{,}\DecValTok{0}\NormalTok{,}\DecValTok{0}\NormalTok{,}\DecValTok{1}\NormalTok{), }\KeywordTok{c}\NormalTok{(}\DecValTok{0}\NormalTok{,}\OperatorTok{-}\DecValTok{1}\NormalTok{,}\DecValTok{1}\NormalTok{,}\DecValTok{0}\NormalTok{), }\KeywordTok{c}\NormalTok{(}\DecValTok{0}\NormalTok{,}\OperatorTok{-}\DecValTok{1}\NormalTok{,}\DecValTok{0}\NormalTok{,}\DecValTok{1}\NormalTok{), }\KeywordTok{c}\NormalTok{(}\DecValTok{0}\NormalTok{,}\DecValTok{0}\NormalTok{,}\OperatorTok{-}\DecValTok{1}\NormalTok{,}\DecValTok{1}\NormalTok{))  }
\NormalTok{s3 <-}\StringTok{ }\KeywordTok{rbind}\NormalTok{(s1, s2)}
\KeywordTok{colnames}\NormalTok{(s3) <-}\KeywordTok{paste}\NormalTok{(}\StringTok{"Coefficient of prog"}\NormalTok{, }\DecValTok{1}\OperatorTok{:}\DecValTok{4}\NormalTok{)}
\KeywordTok{kable}\NormalTok{(s3) }\OperatorTok\StringTok{ }\KeywordTok{kable_styling}\NormalTok{(}\DataTypeTok{full_width =}\NormalTok{ F)}
\end{Highlighting}
\end{Shaded}

\begin{table}[H]
\centering
\begin{tabular}{r|r|r|r}
\hline
Coefficient of prog 1 & Coefficient of prog 2 & Coefficient of prog 3 & Coefficient of prog 4\\
\hline
-1 & 1 & 0 & 0\\
\hline
-1 & 0 & 1 & 0\\
\hline
-1 & 0 & 0 & 1\\
\hline
0 & -1 & 1 & 0\\
\hline
0 & -1 & 0 & 1\\
\hline
0 & 0 & -1 & 1\\
\hline
0 & -1 & -1 & 2\\
\hline
-1 & 0 & -1 & 2\\
\hline
-1 & -1 & 0 & 2\\
\hline
-1 & -1 & 2 & 0\\
\hline
0 & -1 & 2 & -1\\
\hline
-1 & 0 & 2 & -1\\
\hline
-1 & 2 & 0 & -1\\
\hline
-1 & 2 & -1 & 0\\
\hline
0 & 2 & -1 & -1\\
\hline
2 & 0 & -1 & -1\\
\hline
2 & -1 & 0 & -1\\
\hline
2 & -1 & -1 & 0\\
\hline
\end{tabular}
\end{table}

\begin{Shaded}
\begin{Highlighting}[]
\NormalTok{res =}\StringTok{ }\KeywordTok{ScheffeTest}\NormalTok{(model, }\DataTypeTok{conf.level =} \FloatTok{0.95}\NormalTok{, }\DataTypeTok{contrasts =} \KeywordTok{t}\NormalTok{(s3))}
\NormalTok{res[[}\DecValTok{1}\NormalTok{]] =}\StringTok{ }\KeywordTok{cbind}\NormalTok{(res[[}\DecValTok{1}\NormalTok{]], }\DataTypeTok{length =}\NormalTok{ (res[[}\DecValTok{1}\NormalTok{]][,}\StringTok{"upr.ci"}\NormalTok{] }\OperatorTok{-}\StringTok{ }\NormalTok{res[[}\DecValTok{1}\NormalTok{]][,}\StringTok{"lwr.ci"}\NormalTok{]))}
\KeywordTok{print}\NormalTok{(res)}
\end{Highlighting}
\end{Shaded}

\begin{verbatim}

  Posthoc multiple comparisons of means : Scheffe Test 
    95% family-wise confidence level

$variable
                        diff     lwr.ci    upr.ci   pval   length    
prog2-prog1       -2.6666667  -7.674139  2.340806 0.3854 10.01494    
prog3-prog1        1.0000000  -4.007472  6.007472 0.9188 10.01494    
prog4-prog1       -3.0000000  -8.007472  2.007472 0.2967 10.01494    
prog3-prog2        3.6666667  -1.340806  8.674139 0.1682 10.01494    
prog4-prog2       -0.3333333  -5.340806  4.674139 0.9964 10.01494    
prog4-prog3       -4.0000000  -9.007472  1.007472 0.1249 10.01494    
prog4-prog2,prog3 -4.3333333 -13.006530  4.339863 0.4353 17.34639    
prog4-prog1,prog3 -7.0000000 -15.673197  1.673197 0.1204 17.34639    
prog4-prog1,prog2 -3.3333333 -12.006530  5.339863 0.6325 17.34639    
prog3-prog1,prog2  4.6666667  -4.006530 13.339863 0.3774 17.34639    
prog3-prog2,prog4  7.6666667  -1.006530 16.339863 0.0849 17.34639 .  
prog3-prog1,prog4  5.0000000  -3.673197 13.673197 0.3250 17.34639    
prog2-prog1,prog4 -2.3333333 -11.006530  6.339863 0.8286 17.34639    
prog2-prog1,prog3 -6.3333333 -15.006530  2.339863 0.1697 17.34639    
prog2-prog3,prog4 -3.3333333 -12.006530  5.339863 0.6325 17.34639    
prog1-prog3,prog4  2.0000000  -6.673197 10.673197 0.8824 17.34639    
prog1-prog2,prog4  5.6666667  -3.006530 14.339863 0.2368 17.34639    
prog1-prog2,prog3  1.6666667  -7.006530 10.339863 0.9267 17.34639    

---
Signif. codes:  0 '***' 0.001 '**' 0.01 '*' 0.05 '.' 0.1 ' ' 1
\end{verbatim}

\begin{Shaded}
\begin{Highlighting}[]
\NormalTok{model.gh =}\StringTok{ }\KeywordTok{glht}\NormalTok{(model, }\DataTypeTok{linfct =} \KeywordTok{mcp}\NormalTok{(}\DataTypeTok{variable =}\NormalTok{ s3))  }\CommentTok{#set the contrasts}
\NormalTok{res.gh =}\StringTok{ }\KeywordTok{confint}\NormalTok{(model.gh, }\DataTypeTok{test =} \KeywordTok{adjusted}\NormalTok{(}\StringTok{"bonferroni"}\NormalTok{))}
\KeywordTok{print}\NormalTok{(res.gh)}
\end{Highlighting}
\end{Shaded}

\begin{verbatim}

     Simultaneous Confidence Intervals

Multiple Comparisons of Means: User-defined Contrasts


Fit: aov(formula = value ~ variable, data = firmdata)

Quantile = 3.369
95% family-wise confidence level
 

Linear Hypotheses:
        Estimate lwr      upr     
1 == 0   -2.6667  -7.4968   2.1635
2 == 0    1.0000  -3.8301   5.8301
3 == 0   -3.0000  -7.8301   1.8301
4 == 0    3.6667  -1.1635   8.4968
5 == 0   -0.3333  -5.1635   4.4968
6 == 0   -4.0000  -8.8301   0.8301
7 == 0   -4.3333 -12.6994   4.0327
8 == 0   -7.0000 -15.3661   1.3661
9 == 0   -3.3333 -11.6994   5.0327
10 == 0   4.6667  -3.6994  13.0327
11 == 0   7.6667  -0.6994  16.0327
12 == 0   5.0000  -3.3661  13.3661
13 == 0  -2.3333 -10.6994   6.0327
14 == 0  -6.3333 -14.6994   2.0327
15 == 0  -3.3333 -11.6994   5.0327
16 == 0   2.0000  -6.3661  10.3661
17 == 0   5.6667  -2.6994  14.0327
18 == 0   1.6667  -6.6994  10.0327
\end{verbatim}

\begin{Shaded}
\begin{Highlighting}[]
\NormalTok{res.gh}\OperatorTok{$}\NormalTok{confint[,}\StringTok{"upr"}\NormalTok{] }\OperatorTok{-}\StringTok{ }\NormalTok{res.gh}\OperatorTok{$}\NormalTok{confint[,}\StringTok{"lwr"}\NormalTok{] }\CommentTok{#compute the lengths of CI's}
\end{Highlighting}
\end{Shaded}

\begin{verbatim}
        1         2         3         4         5         6         7 
 9.660288  9.660288  9.660288  9.660288  9.660288  9.660288 16.732110 
        8         9        10        11        12        13        14 
16.732110 16.732110 16.732110 16.732110 16.732110 16.732110 16.732110 
       15        16        17        18 
16.732110 16.732110 16.732110 16.732110 
\end{verbatim}

\textbf{For each of the three sets, which procedure works out the best?}

The discussion regarding the best procedure has been divided into the
three parts, each for different sets of contrasts.

\begin{enumerate}
\def\labelenumi{\arabic{enumi}.}
\item
  Starting with S1, Tukey's Honest Significant Difference method gives a
  95\% confidence interval of length 9.18 units, Scheffe's method gives
  a 95\% CI of length 10.01 units and Bonferroni's method gives a 95\%
  CI of length 9.97 units. Clearly, from the tightness of the confidence
  interval as the evaluation criterion, it is seen that Tukey's method
  works best for pairwise contrasts.
\item
  For the set of contrasts in S2, Scheffe's method gives the 95\%
  confidence interval of length 17.34 units, while Bonferroni's method
  gives the 95\% CI of length 16.52 units. In this regard, Bonferroni's
  method works better.
\item
  For the set of contrasts in S2, Scheffe's method gives the 95\%
  confidence interval of length 10.01 units for the pairwise difference
  contrasts and 17.34 units for the general contrasts, while
  Bonferroni's method gives the 95\% CI of length 9.66 units for
  pairwise difference contrasts and 16.73 units for the general
  contrasts. In this case also, Bonferroni's method works better.
\end{enumerate}

Finally, we see that Tukey's test is best in case of finding the
simulteneous confidence intervals pairwise difference contrasts. But, in
case of sets of more general contrasts, bonferroni's method works
better. However, observe that the length of the confidence interval in
scheffe's method does not increase even if more contrasts are added,
while the length of CI given by bonferroni's method increases. This
shows an indication that if there are lots of contrasts for which
simulteneous confidence intervals are to be found, bonferroni's method
will not remain better than Scheffe's method.

\section{THANK YOU}\label{thank-you}


\end{document}
